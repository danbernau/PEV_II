\input cwebmac






\font\sixteenrm =cmr10  scaled 1600  % fifteen point roman
\font\ninerm    =cmr10  scaled  900  % nine point roman

\datethis
\null
\bigskip
\centerline{\sixteenrm Source Terminal Network Unreliability Estimation}
\smallskip
\centerline{\sixteenrm by Crude Monte Carlo (1)}
\smallskip
\centerline{v1.1}
\smallskip
\centerline{\ninerm Leslie Murray}
\vskip 0pt
\centerline{\ninerm Universidad Nacional de Rosario}
\bigskip

An undirected graph $G=(V,E)$, where $V$ is a set of $n$ absolutely reliable
nodes and $E$ a set of $m$ independent unreliable links, is a frequently used
model for studying communication network reliability. Being $X_i=1$ when the
$i^{th}$ link is {\it operative} and $X_i=0$ when the $i^{th}$ link is
{\it failed}, vector ${\bf X}=(X_1,X_2,\cdots,X_m)$ denotes the state of the
links and, thereby, the state of the whole network. The single link reliability
is defined as $r_i=P[X_i=1]$ whereas the single link unreliability as
$q_i=1-r_i=P[X_i=0]$.

$\Phi({\bf X})$ is called structure function, a function that equals $1$ when
the network is {\it operative} and $0$ when the network is {\it failed}. In
this program $\Phi({\bf X})$ is associated to the source--terminal network
reliability model in which the network is considered {\it operative} if there
is a path of {\it operative} links between two nodes $s$ and $t$, and
{\it failed} if there is no path of {\it operative} links between nodes $s$ and
$t$. By means of this function, the network reliability $R$ and unreliability
$Q$ are defined, respectively, as $R=P[\Phi({\bf X})=1]$ and
$Q=P[\Phi({\bf X})=0]$.

Straightforward (also called Standard, Direct or Crude) Monte Carlo estimations
of the network reliability and unreliability can be computed, respectively, as
$\widehat{R} = 1/N \sum_{i=1}^{N} \Phi({\bf X}^{(i)})$ and $\widehat{Q} = 1/N
\sum_{i=1}^{N}(1-\Phi({\bf X}^{(i)})$ where ${\bf X}^{(i)},\ i=1,\cdots,N$ are
{\it i.i.d.} samples taken from $f_{{\bf X}}({\bf x})$: the probability mass
function of vector ${\bf X}$. For highly reliable networks most of the
$X^{(i)}$
samples will equal $1$ and, as a consequence, most of the replications
$\Phi({\bf X}^{(i)})$ will equal $1$ also. For extremely reliable networks,
with
unreliabilities in the order of $10^{-10}$ or even less, it will take an
average
of $10^{10}$ replications to get $\Phi({\bf X}^{(i)})=0$ at least once. For
these networks $\Phi({\bf X})=0$ is a rare event.

The accuracy of the unreliability estimation can be assessed by means of a
relative error defined as $V\{\widehat{Q}\}^{1/2}/E\{\widehat{Q}\}$, that in
the
case of Crude Monte Carlo takes the form $((1-Q)/NQ)^{1/2}\approx(1/NQ)^{1/2}$.
This shows a weakness of the method for the case of highly reliable networks
($Q
$ very low) and the reason why accurate estimations require a high number of
replications ($N$ very high).

As for the program $E\{\widehat{Q}\}$ and $V\{\widehat{Q}\}$ are both unknown,
the following unmbiased estimators are calculated, instead:
\smallskip
\centerline{$1/N\sum_{i=1}^{N}(1-\Phi({\bf X}^{(i)})$ \ \ \ for \ \ \ $E
\{\widehat{Q}\}$}
\smallskip
\centerline{$\sum_{i=1}^{N}(1-\Phi({\bf X}^{(i)})^2/(N(N-1))-\left(\sum_{i=1}^
{N}(1-\Phi({\bf X}^{(i)})\right)^2/(N-1)$ \ \ \ for \ \ \ $V\{\widehat{Q}\}$}
\smallskip
This program attempts to obtain these estimators and, by means of them, the
relative error too. The network under study is passed to the program as a text
file with the following format:
\medskip
\vbox{
$<${\bf node} {\it s}$>$

$<${\bf node} {\it t}$>$

$<${\it n}$>$

$<${\it l}$>$

\settabs\+\indent
& $<${\bf node} {\it n}$>$\quad & $<adj_n>$\quad & $<${\bf node} dest$>$\quad &
$<${\it rlb} $>$\quad & $<${\bf node} dest$>$\quad & $<${\it rlb}$>$\quad &
$<${
\bf node} dest$>$\quad \cr

\+& $<${\bf node} 1$>$ & $<adj_1>$ & $<${\bf node} dest$>$ & $<${\it rlb}$>
$ & $<${\bf node} dest$>$ & $<${\it rlb}$>$ & $\cdots$\quad \cr

\+& $<${\bf node} 2$>$ & $<adj_2>$ & $<${\bf node} dest$>$ & $<${\it rlb}$>
$ & $<${\bf node} dest$>$ & $<${\it rlb}$>$ & $\cdots$\quad \cr

\+& $<${\bf node} 3$>$ & $<adj_3>$ & $<${\bf node} dest$>$ & $<${\it rlb}$>
$ & $<${\bf node} dest$>$ & $<${\it rlb}$>$ & $\cdots$\quad \cr

\+& \ \ \ \ \ \ \ $\vdots$ & \ \ \ \ \ \ $\vdots$ & \ \ \ \ \ \ \ $\vdots$
& \ \ \ \ $\vdots$ & $\ \ \ \ \ \ \ \vdots$ & \ \ \ \ $\vdots$ \cr

\+& $<${\bf node} {\it n}$>$ & $<adj_n>$ & $<${\bf node} dest$>$ & $<${\it
rlb}$>$ & $<${\bf node} dest$>$ & $<${\it rlb}$>$ & $\cdots$\quad \cr

}
\bigskip
All expressions like $<\ldots>$, are integers numbers. $<${\bf node} {\it s}$>$
and $<${\bf node} {\it t}$>$ are, respectively, the {\it source} and
{\it terminal} nodes (to estimate the source--terminal unreliability).
$<${\it n}$>$ is the number of {\it nodes} and $<${\it l}$>$ the number of {\it
links}. Every one of the subsequent lines have the following meaning: $<${\bf
node} $i>$ has a number $<adj_i>$ of adjacent nodes, each one of them
identified
by the correponding number ($<${\bf node} dest$>$) followed by its single
reliability value ($<${\it rlb}$>$).




\N{1}{1}The program general structure. The whole program is shown as a set of
sections, each one of them is introduced in the following items.

\Y\B\X2:Library Headers and Included Files\X;\6
\X4:New Type Definition and Global Variables\X;\6
\X5:Prototypes of Auxiliary Functions\X;\7
\&{int} \\{main}(\&{int} \\{argc}${},\39{}$\&{char} ${}{*}\\{argv}[\,]){}$\1\1%
\2\2\6
${}\{{}$\1\6
\X6:Local Variables of \PB{\\{main}(\,)}\X;\6
\X7:Input Data Validation\X;\6
\X8:Crude Monte Carlo Algorithm\X;\6
\X9:Print of Output\X;\6
\&{return} \T{0};\6
\4${}\}{}$\2\7
\X10:Auxiliary Functions\X;\par
\fi

\M{2}
At the top, the inclusions are as usual, mostly to allow the use of
input-output and mathematical functions. The most remarkable files are the
library {\tt time.h} to make use of functions to measure the execution time and
{\tt mt19937ar.c}, the code of the Mersenne Twister random numbers generator.
\Y\B\4\X2:Library Headers and Included Files\X${}\E{}$\6
\8\#\&{include} \.{<stdio.h>}\6
\8\#\&{include} \.{<stdlib.h>}\6
\8\#\&{include} \.{<math.h>}\6
\8\#\&{include} \.{<time.h>}\6
\8\#\&{include} \.{"mt19937ar.c"}\par
\U1.\fi

\M{3}
For clarification, single random numbers from the Mersenne Twister are called
\PB{\|U}.
\Y\B\4\D$\|U$ \5
\\{genrand\_real1}(\,)\SHC{ a single random number (from the Mersenne Twister)
}\par
\fi

\M{4}
Some structures are to be allocated: one unit of {\bf struct link} for every
link and one unit of {\bf struct network} for the whole network. In the input
file from which the network topolgy is loaded, the nodes are enumerated from 1
to \PB{\\{numnodes}}. The adjacency list of the graph (network) makes use of
these
numbers. Such adjacency lists are rebuilt every time a new state is sampled for
every link, and the latest version of the adjacency lists is used by the
algorithms to determine whether nodes \PB{\|s} and \PB{\|t} are connected.
\Y\B\4\X4:New Type Definition and Global Variables\X${}\E{}$\6
\&{typedef} \&{struct} \&{link} ${}\{{}$\1\6
\&{int} \\{adj};\SHC{1 or 0 to build the adjacency martix of the network
 }\6
\&{double} \\{rlb};\SHC{single link reliability
 }\6
\&{int} \\{smp};\SHC{1 or 0 to build the adjacency matrix of the network, after
random fail
 }\2\6
${}\}{}$ \&{lnk}${},{}$  ${}{*}\&{pt\_lnk};{}$\6
\&{typedef} \&{struct} \&{network} ${}\{{}$\1\6
\&{int} \|s;\SHC{source node
 }\6
\&{int} \|t;\SHC{target node
 }\6
\&{int} \\{numnodes};\SHC{number of nodes
 }\6
\&{int} \\{numlinks};\SHC{number of links
 }\6
\&{struct} \&{link} ${}{*}{*}\|I{}$;\SHC{matrix of links, size:
I[numnodes+1][numnodes+1] (I[0][0], unused)
 }\2\6
${}\}{}$ \&{net}${},{}$  ${}{*}\&{pt\_net};{}$\6
\&{int} ${}{*}{*}\\{list}{}$;\SHC{matrix to perform as "list" of "adjacency
lists" }\6
\&{int} ${}{*}\\{visited}{}$;\SHC{array to keep track of the visited nodes in
the DFS
 }\6
\&{int} \\{connected};\SHC{1 if there is a path connecting "s" and "t" and 0
otherwise
 }\6
\&{int} \\{seed};\SHC{seed for the random numbers generator Mersenne Twister
 }%
\par
\U1.\fi

\M{5}
All functions and algorithms deal with a network whose topolgy is passed to the
program as the argument \PB{\\{argv}[\T{1}]} of the \PB{\\{main}(\,)}. This
argument is the name
of a file that is finally passed to the function \PB{\\{Initialize}} as the
argument
\PB{\\{filename}}. Function \PB{\\{Initialize}} allocates one unit of a {\bf
struct network}
and returns a pointer to it. This pointer is taken as the argument by all the
other functions, namely: \PB{\|X(\,)}, \PB{\\{Fail}(\,)}, \PB{\.{DFS}(\,)} and %
\PB{\\{Phi}(\,)}. All these
functions are thought to operate this way: \PB{\|X(\,)} sets a random state on
every
link by means of a numerical value, if 0, the link is removed from the network,
if 1, the link remains untouched; \PB{\\{Fail}(\,)} sets a random state on
every link;
\PB{\.{DFS}(\,)} performs a Depth First Search starting from node \PB{\|s} and
tells \PB{\\{Phi}(\,)}
whether node \PB{\|t} is reached or not, if node \PB{\|t} is reached \PB{%
\\{Phi}(\,)} returns 1,
otherwise it returns 0.
\Y\B\4\X5:Prototypes of Auxiliary Functions\X${}\E{}$\6
\&{pt\_net} \\{Initialize}(\&{char} ${}{*}\\{filename}){}$;\SHC{allocates a
network associated to the info in file \PB{\\{filename}}
 }\6
\&{int} \|X(\&{int} \\{node1}${},\39{}$\&{int} \\{node2}${},\39{}$\&{pt\_net} %
\\{nt});\SHC{returns 1 if link \PB{\\{node1}}--\PB{\\{node2}} is operative, 0
otherwise
 }\6
\&{void} \\{Fail}(\&{pt\_net} \\{nt});\SHC{set link random fails and builds the
adjacency lists after that fail
 }\6
\&{void} \.{DFS}(\&{int} \\{node}${},\39{}$\&{pt\_net} \\{nt});\SHC{performs
DFS from node \PB{\|s}, stops when node \PB{\|t} is reached
 }\6
\&{int} \\{Phi}(\&{pt\_net} \\{nt});\SHC{returns 1 if \PB{\|s} and \PB{\|t} are
connected and 0 otherwise
 }\par
\U1.\fi

\M{6}
Function \PB{\\{main}(\,)} makes use of many local variables, none of which
deserve a
particular explanation. Most of them operate as indexes and as memory units to
retain some values during mathematical calculation. \PB{\|n} is a pointer to
the
network under study. \PB{\\{ti}} receives the value of function \PB{\\{clock}(%
\,)} (library
{\tt time.h}). Function \PB{\\{clock}(\,)} returns the elapsed time since the
beginning
of the execution.
\Y\B\4\X6:Local Variables of \PB{\\{main}(\,)}\X${}\E{}$\6
\&{int} \|i${},{}$ \|j${},{}$ \|k${},{}$ \|S${},{}$ \\{size};\6
\&{double} \|X${},{}$ \|V${},{}$ \|Q${},{}$ \|t;\6
\&{pt\_net} \|n;\6
\&{clock\_t} \\{ti};\par
\U1.\fi

\M{7}
The program accepts input data at the command line. If compilation is such that
{\tt crude1} is the name of the executable, the program runs as:
\smallskip
\centerline{\tt ./crude1 <File> <Size> <Seed>}
\smallskip
where {\tt<File>} is the text file with the network topolgy info, {\tt<Size>}
the number of Monte Carlo trials and {\tt<Seed>} the value to set the starting
point of the random numbers generator (Mersenne Twister). Some validation on
these data is aimed to check: the number of inputs to be not less, and not more
than three, the value of {\tt<Size>} expecting that is not less than one and
the
{\tt<Seed>}, restricting it to positive values.
\Y\B\4\X7:Input Data Validation\X${}\E{}$\6
\&{if} ${}(\\{argc}<\T{4}){}$\5
${}\{{}$\1\6
\\{printf}(\.{"\\n\ Some\ input\ data\ }\)\.{is\ missing!\ Run\ as:\\}\)%
\.{n"});\6
\\{printf}(\.{"\\n\ ./executable\ <Fi}\)\.{le>\ <Size>\ <Seed>\\n\\}\)\.{n"});\6
\\{exit}(\T{1});\6
\4${}\}{}$\2\6
\&{if} ${}(\\{argc}>\T{4}){}$\5
${}\{{}$\1\6
\\{printf}(\.{"\\n\ You've\ entered\ m}\)\.{ore\ data\ than\ necess}\)\.{ary!\
Run\ as:\\n\\n"});\6
\\{printf}(\.{"\\n\ ./executable\ <Fi}\)\.{le>\ <Size>\ <Seed>\\n\\}\)\.{n"});\6
\\{exit}(\T{1});\6
\4${}\}{}$\2\6
\&{if} ${}((\\{size}\K\\{atoi}(\\{argv}[\T{2}]))<\T{1}){}$\5
${}\{{}$\1\6
\\{printf}(\.{"\\n\ The\ number\ of\ tr}\)\.{ials\ can\ not\ be\ less}\)\.{\
than\ 1!\ Run\ as:\\n"});\6
\\{printf}(\.{"\\n\ ./executable\ <Fi}\)\.{le>\ <Size>\ <Seed>\\n\\}\)\.{n"});\6
\\{exit}(\T{1});\6
\4${}\}{}$\2\6
\&{if} ${}((\\{seed}\K\\{atoi}(\\{argv}[\T{3}]))<\T{0}){}$\5
${}\{{}$\1\6
\\{printf}(\.{"\\n\ The\ seed\ can\ not}\)\.{\ be\ negative!\ Run\ as}\)\.{:%
\\n"});\6
\\{printf}(\.{"\\n\ ./executable\ <Fi}\)\.{le>\ <Size>\ <Seed>\\n\\}\)\.{n"});\6
\\{exit}(\T{1});\6
\4${}\}{}$\2\par
\U1.\fi

\M{8}
As explained in the introduction, the core of this program is a very simple
algorithm, aimed to repeat a number \PB{\\{size}} of times a cycle in which
function
\PB{\\{Fail}(\,)} sets a random value of either 0 or 1 to every link; after
this,
function \PB{\\{Phi}(\,)} returns 1 if nodes \PB{\|s} and \PB{\|t} are
connected and 0 otherwise.
Accumulation of the complement of the value of \PB{\\{Phi}(\,)}, and its
square, let
the estimation of the unreliability \PB{\|Q} and its corresponding variance %
\PB{\|V} be
done. \PB{\\{ti}} saves the value of the function \PB{\\{clock}(\,)} just
before the algorithm
starts, and after completion of the run \PB{\\{ti}} is subtracted from the
current
value of \PB{\\{clock}(\,)}. Such difference is the execution time of the
algorithm (note
that the initialization process time is not considered).
\Y\B\4\X8:Crude Monte Carlo Algorithm\X${}\E{}$\6
$\|n\K\\{Initialize}(\\{argv}[\T{1}]);{}$\6
${}\\{ti}\K\\{clock}(\,);{}$\6
${}\|S\K\T{0};{}$\6
${}\|X\K\T{0.0};{}$\6
${}\|V\K\T{0.0};{}$\6
\&{for} ${}(\|k\K\T{0};{}$ ${}\|k<\\{size};{}$ ${}\|k\PP){}$\5
${}\{{}$\1\6
\\{Fail}(\|n);\6
${}\|X\K(\T{1}-\\{Phi}(\|n));{}$\6
${}\|S\MRL{+{\K}}\|X;{}$\6
${}\|V\MRL{+{\K}}\|X*\|X;{}$\6
\4${}\}{}$\2\6
${}\|Q\K{}$(\&{double}) \|S${}/\\{size};{}$\6
${}\|V\K(\|V/\\{size}-\|Q*\|Q)/(\\{size}-\T{1});{}$\6
${}\|t\K(\&{double})(\\{clock}(\,)-\\{ti})/\.{CLOCKS\_PER\_SEC}{}$;\par
\U1.\fi

\M{9}
The reason why this version is called ``{\tt(1)}'' will be revealed with the
advent of versions {\tt(2)} and {\tt(3)}$\ldots$ At the moment it's enough to
say that this is the simplest version (at least the simplest out of the three)
to implement the basis of the Crude Monte Carlo algorithm for the estimation of
network unreliability, \PB{\|Q}. The main output is therefore the value of \PB{%
\|Q}. This
value is shown toghether with the name of the network under study, the number
of
Monte Carlo replications, the execution time in seconds, the variance \PB{\|V},
the
standard deviation $V^{1/2}$ and the relative error $V^{1/2}/Q$.
\Y\B\4\X9:Print of Output\X${}\E{}$\6
$\\{printf}(\.{"\\n\ \ Network:\ \%s\ \ \ R}\)\.{eplications:\ \%d\ \ \ Ex}\)%
\.{ecTime=\%f"},\39\\{argv}[\T{1}],\39\\{size},\39\|t);{}$\6
\\{printf}(\.{"\\n\ \ ***************}\)\.{****\ CRUDE\ MONTE\ CAR}\)\.{LO\ (1)%
\ *************}\)\.{*****"});\6
${}\\{printf}(\.{"\\n\ \ \ \ \ \ \ Unreliabil}\)\.{ity\ \ \ \ Q\ =\ \%1.16f\ =\
}\)\.{\%1.2e"},\39\|Q,\39\|Q);{}$\6
${}\\{printf}(\.{"\\n\ \ \ \ \ \ \ Variance\ \ }\)\.{\ \ \ \ \ \ \ V\ =\ %
\%1.16f\ =\ }\)\.{\%1.2e"},\39\|V,\39\|V);{}$\6
${}\\{printf}(\.{"\\n\ \ \ \ \ \ \ Std.\ Dev.\ }\)\.{\ \ \ \ \ \ SD\ =\ \%1.16f%
\ =\ }\)\.{\%1.2e"},\39\\{sqrt}(\|V),\39\\{sqrt}(\|V));{}$\6
${}\\{printf}(\.{"\\n\ \ \ \ \ \ \ Relative\ E}\)\.{rror\ \ RE\ =\ \%1.16f\ =\
}\)\.{\%1.2f\%\%"},\39\\{sqrt}(\|V)/\|Q,\39\T{100}*\\{sqrt}(\|V)/\|Q);{}$\6
\\{printf}(\.{"\\n\ \ ***************}\)\.{********************}\)%
\.{********************}\)\.{*****\\n\\n"});\par
\U1.\fi

\N{1}{10}The set of Auxiliary Functions. These functions provide support to
implement
the main operations required by the program. They are shown here, classified in
three sections, each one of them containing code clearly associated to the name
given to it.
\Y\B\4\X10:Auxiliary Functions\X${}\E{}$\6
\X11:Initialization\X;\6
\X12:Fail Generation\X;\6
\X13:Function of Structure\X;\par
\U1.\fi

\M{11}
Function \PB{\\{Initialize}(\,)} builds up and allocates the network data
structure
\PB{\&{net}} with information read from the file that holds the network data
(\PB{\\{filename}}). It also initializes the random numbers generator Mersenne
Twister.
\Y\B\4\X11:Initialization\X${}\E{}$\6
\&{pt\_net} \\{Initialize}(\&{char} ${}{*}\\{filename}){}$\1\1\2\2\6
${}\{{}$\1\6
\&{pt\_net} \\{pt\_n};\6
\&{FILE} ${}{*}\\{fp};{}$\6
\&{int} \|i${},{}$ \|j${},{}$ \\{node1}${},{}$ \\{node2}${},{}$ \\{num};\6
\&{double} \\{reliability};\6
\,\SHC{Allocate one unit of the structure \PB{\&{net}} to hold the network
info
 }\7
\&{if} ${}((\\{pt\_n}\K{}$(\&{pt\_net}) \\{calloc}${}(\T{1},\39\&{sizeof}(%
\&{net})))\E\NULL){}$\5
${}\{{}$\1\6
\\{printf}(\.{"\\n\ Fail\ attempting\ }\)\.{to\ allocate\ memory..}\)\.{.%
\\n"});\6
\\{exit}(\T{1});\6
\4${}\}{}$\6
\,\SHC{Open the file with the network info and scan for the source and target
}\6
\,\SHC{node, the number of nodes and the number of links at the top of it
 }\2\6
\&{if} ${}((\\{fp}\K\\{fopen}(\\{filename},\39\.{"r"}))\E\NULL){}$\5
${}\{{}$\1\6
\\{printf}(\.{"\\n\ Fail\ attempting\ }\)\.{to\ open\ a\ disk\ file.}\)\.{..%
\\n"});\6
\\{exit}(\T{1});\6
\4${}\}{}$\2\6
${}\\{fscanf}(\\{fp},\39\.{"\%d"},\39{\AND}\\{pt\_n}\MG\|s);{}$\6
${}\\{fscanf}(\\{fp},\39\.{"\%d"},\39{\AND}\\{pt\_n}\MG\|t);{}$\6
${}\\{fscanf}(\\{fp},\39\.{"\%d"},\39{\AND}\\{pt\_n}\MG\\{numnodes});{}$\6
${}\\{fscanf}(\\{fp},\39\.{"\%d"},\39{\AND}\\{pt\_n}\MG\\{numlinks}){}$;%
\SHC{Allocate matrix I with a size of (numnodes+1)*(numnodes+1), and link it
to
 }\SHC{the corresponding pointer of network net
 }\6
\&{if} ${}((\\{pt\_n}\MG\|I\K{}$(\&{pt\_lnk} ${}{*}){}$ \\{calloc}${}(\\{pt\_n}%
\MG\\{numnodes}+\T{1},\39{}$\&{sizeof}(\&{struct} \&{link} ${}{*})))\E\NULL){}$%
\5
${}\{{}$\1\6
\\{printf}(\.{"\\n\ Fail\ attempting\ }\)\.{to\ allocate\ memory..}\)\.{.%
\\n"});\6
\\{exit}(\T{1});\6
\4${}\}{}$\2\6
\&{for} ${}(\|i\K\T{0};{}$ ${}\|i\Z\\{pt\_n}\MG\\{numnodes};{}$ ${}\|i\PP){}$\5
${}\{{}$\1\6
\&{if} ${}((\\{pt\_n}\MG\|I[\|i]\K{}$(\&{pt\_lnk}) \\{calloc}${}(\\{pt\_n}\MG%
\\{numnodes}+\T{1},\39{}$\&{sizeof}(\&{struct} \&{link})))${}\E\NULL){}$\5
${}\{{}$\1\6
\\{printf}(\.{"\\n\ Fail\ attempting\ }\)\.{to\ allocate\ memory..}\)\.{.%
\\n"});\6
\\{exit}(\T{1});\6
\4${}\}{}$\2\6
\4${}\}{}$\6
\,\SHC{Initialize matrix I with 0s for the adjacencies and 0.0s for the
 }\6
\,\SHC{reliabilities of every link
 }\2\6
\&{for} ${}(\|i\K\T{1};{}$ ${}\|i\Z\\{pt\_n}\MG\\{numnodes};{}$ ${}\|i\PP){}$\1%
\6
\&{for} ${}(\|j\K\T{1};{}$ ${}\|j\Z\\{pt\_n}\MG\\{numnodes};{}$ ${}\|j\PP){}$\5
${}\{{}$\1\6
${}\\{pt\_n}\MG\|I[\|i][\|j].\\{adj}\K\T{0};{}$\6
${}\\{pt\_n}\MG\|I[\|i][\|j].\\{rlb}\K\T{0.0};{}$\6
${}\\{pt\_n}\MG\|I[\|i][\|j].\\{smp}\K\T{0};{}$\6
\4${}\}{}$\6
\,\SHC{Load matrix I values from the file with the network info and set the
 }\6
\,\SHC{corresponding values of adjacency and reliability
 }\2\2\6
\&{for} ${}(\|i\K\T{1};{}$ ${}\|i\Z\\{pt\_n}\MG\\{numnodes};{}$ ${}\|i\PP){}$\5
${}\{{}$\1\6
${}\\{fscanf}(\\{fp},\39\.{"\%d\%d"},\39{\AND}\\{node1},\39{\AND}\\{num});{}$\6
\&{for} ${}(\|j\K\T{1};{}$ ${}\|j\Z\\{num};{}$ ${}\|j\PP){}$\5
${}\{{}$\1\6
${}\\{fscanf}(\\{fp},\39\.{"\%d\%lf"},\39{\AND}\\{node2},\39{\AND}%
\\{reliability});{}$\6
${}\\{pt\_n}\MG\|I[\\{node1}][\\{node2}].\\{adj}\K\T{1};{}$\6
${}\\{pt\_n}\MG\|I[\\{node1}][\\{node2}].\\{rlb}\K\\{reliability};{}$\6
\4${}\}{}$\2\6
\4${}\}{}$\2\6
\\{fclose}(\\{fp});\6
\,\SHC{Take the \PB{\\{seed}} value and initialize the pseudorandom numbers
generator
 }\6
\,\SHC{Mersenne Twister
 }\6
\\{init\_genrand}(\\{seed});\6
\,\SHC{Allocate matrix \PB{\\{list}} of size (numnodes+1)*(numnodes+1), same as
matrix I
 }\6
\&{if} ${}((\\{list}\K{}$(\&{int} ${}{*}{*}){}$ \\{calloc}${}(\\{pt\_n}\MG%
\\{numnodes}+\T{1},\39{}$\&{sizeof}(\&{int} ${}{*})))\E\NULL){}$\5
${}\{{}$\1\6
\\{printf}(\.{"\\n\ Fail\ attempting\ }\)\.{to\ allocate\ memory..}\)\.{.%
\\n"});\6
\\{exit}(\T{1});\6
\4${}\}{}$\2\6
\&{for} ${}(\|i\K\T{0};{}$ ${}\|i\Z\\{pt\_n}\MG\\{numnodes};{}$ ${}\|i\PP){}$\5
${}\{{}$\1\6
\&{if} ${}((\\{list}[\|i]\K{}$(\&{int} ${}{*}){}$ \\{calloc}${}(\\{pt\_n}\MG%
\\{numnodes},\39\&{sizeof}(\&{int})))\E\NULL){}$\5
${}\{{}$\1\6
\\{printf}(\.{"\\n\ Fail\ attempting\ }\)\.{to\ allocate\ memory..}\)\.{.%
\\n"});\6
\\{exit}(\T{1});\6
\4${}\}{}$\2\6
\4${}\}{}$\6
\,\SHC{Initialize matrix \PB{\\{list}}, filling it with 0s
 }\2\6
\&{for} ${}(\|i\K\T{1};{}$ ${}\|i\Z\\{pt\_n}\MG\\{numnodes};{}$ ${}\|i\PP){}$\1%
\6
\&{for} ${}(\|j\K\T{1};{}$ ${}\|j\Z\\{pt\_n}\MG\\{numnodes};{}$ ${}\|j\PP){}$\5
${}\{{}$\1\6
${}\\{list}[\|i][\|j]\K\T{0};{}$\6
\4${}\}{}$\6
\,\SHC{Allocate the \PB{\\{visited}} array and set 0 for all nodes (not
visited)
 }\2\2\6
\&{if} ${}((\\{visited}\K{}$(\&{int} ${}{*}){}$ \\{calloc}${}(\\{pt\_n}\MG%
\\{numnodes},\39\&{sizeof}(\&{int})))\E\NULL){}$\5
${}\{{}$\1\6
\\{printf}(\.{"\\n\ Fail\ attempting\ }\)\.{to\ allocate\ memory..}\)\.{.%
\\n"});\6
\\{exit}(\T{1});\6
\4${}\}{}$\2\6
\&{for} ${}(\|i\K\T{1};{}$ ${}\|i\Z\\{pt\_n}\MG\\{numnodes};{}$ ${}\|i\PP){}$\1%
\5
${}\\{visited}[\|i]\K\T{0};{}$\2\6
${}\\{connected}\K\T{0};{}$\6
\&{return} \\{pt\_n};\6
\4${}\}{}$\2\par
\U10.\fi

\M{12}
Depending on the probability distribution of the link between \PB{\\{node}%
\T{1}} and
\PB{\\{node}\T{2}}, function \PB{\|X(\,)} returns 1 if such link is operative
and 0 otherwise. It
is accepted that function \PB{\|X(\,)} is only required for links that
actuallly exist,
i.e. links for which \PB{$\|I[\\{node1}][\\{node2}].\\{adj}\K\T{1}$}.

Function \PB{\\{Fail}(\,)} sets a random value of 0 or 1 in \PB{\\{smp}} for
every existing link
of network \PB{\\{nt}}. This way, a randomly failed network is built up and
saved
into matrix \PB{$\|I[\,][\,].\\{smp}$} and also in \PB{\\{list}[\,][\,]}. It is
accepeted that the network
graph is undirected and that the reliability of every link has the same value
in
both directions. It is therefore unnecesary to sample all the matrix elements,
it suffices to do it for all the elements above the diagonal and to set the
same
value to the simetric element (the diagonal elements are all 0).

\Y\B\4\X12:Fail Generation\X${}\E{}$\6
\&{int} \|X(\&{int} \\{node1}${},\39{}$\&{int} \\{node2}${},\39{}$\&{pt\_net} %
\\{nt})\1\1\2\2\6
${}\{{}$\1\6
\&{if} ${}(\|U<\\{nt}\MG\|I[\\{node1}][\\{node2}].\\{rlb}){}$\1\5
\&{return} \T{1};\2\6
\&{else}\1\5
\&{return} \T{0};\2\6
\4${}\}{}$\2\7
\&{void} \\{Fail}(\&{pt\_net} \\{nt})\1\1\2\2\6
${}\{{}$\1\6
\&{int} \|i${},{}$ \|j${},{}$ \|k;\7
\&{for} ${}(\|i\K\T{1};{}$ ${}\|i\Z\\{nt}\MG\\{numnodes};{}$ ${}\|i\PP){}$\1\6
\&{for} ${}(\|j\K\|i+\T{1};{}$ ${}\|j\Z\\{nt}\MG\\{numnodes};{}$ ${}\|j\PP){}$%
\1\6
\&{if} ${}(\\{nt}\MG\|I[\|i][\|j].\\{adj}){}$\5
${}\{{}$\1\6
${}\\{nt}\MG\|I[\|i][\|j].\\{smp}\K(\\{nt}\MG\|I[\|i][\|j].\\{adj}\W\|X(\|i,\39%
\|j,\39\\{nt}));{}$\6
${}\\{nt}\MG\|I[\|j][\|i].\\{smp}\K\\{nt}\MG\|I[\|i][\|j].\\{smp};{}$\6
\4${}\}{}$\2\2\2\6
\&{for} ${}(\|i\K\T{1};{}$ ${}\|i\Z\\{nt}\MG\\{numnodes};{}$ ${}\|i\PP){}$\5
${}\{{}$\SHC{Clean the prior adjacency lists
 }\1\6
${}\|k\K\T{1};{}$\6
\&{while} ${}(\\{list}[\|i][\|k]>\T{0}){}$\5
${}\{{}$\1\6
${}\\{list}[\|i][\|k\PP]\K\T{0};{}$\6
\4${}\}{}$\2\6
\4${}\}{}$\2\6
\&{for} ${}(\|i\K\T{1};{}$ ${}\|i\Z\\{nt}\MG\\{numnodes};{}$ ${}\|i\PP){}$\5
${}\{{}$\SHC{Create the new adjacency lists
 }\1\6
${}\|k\K\T{1};{}$\6
\&{for} ${}(\|j\K\T{1};{}$ ${}\|j\Z\\{nt}\MG\\{numnodes};{}$ ${}\|j\PP){}$\5
${}\{{}$\1\6
\&{if} ${}(\\{nt}\MG\|I[\|i][\|j].\\{smp}){}$\5
${}\{{}$\1\6
${}\\{list}[\|i][\|k\PP]\K\|j;{}$\6
\4${}\}{}$\2\6
\4${}\}{}$\2\6
\4${}\}{}$\2\6
\&{return};\6
\4${}\}{}$\2\par
\U10.\fi

\M{13}
Functions \PB{\\{Phi}(\,)} and \PB{\.{DFS}(\,)} both update the value of
variable \PB{\\{connected}}.
\PB{\\{Phi}(\,)} sets it to 0 (as an indication that there is no path of
operative links
between node \PB{\|s} and \PB{\|t}). It also assumes that no node has been
visited by the
Depth First Search yet. Then it calls \PB{\.{DFS}(\,)} that, starting frome
node \PB{\|s} sets
\PB{\\{connected}} to 1 and stop if there is a path of operative links between
nodes
\PB{\|s} and \PB{\|t}, and does nothing if there is not such path.
\Y\B\4\X13:Function of Structure\X${}\E{}$\6
\&{int} \\{Phi}(\&{pt\_net} \\{nt})\1\1\2\2\6
${}\{{}$\1\6
\&{int} \|k;\7
${}\\{connected}\K\T{0};{}$\6
\&{for} ${}(\|k\K\T{1};{}$ ${}\|k\Z\\{nt}\MG\\{numnodes};{}$ ${}\|k\PP){}$\1\5
${}\\{visited}[\|k]\K\T{0}{}$;\6
\,\SHC{DFS() makes a Depth First Search from node \PB{\|s} and:
 }\6
\,\SHC{- returns 1 if node \PB{\|t} is reached
 }\6
\,\SHC{- returns 0 if node \PB{\|t} is not reached
 }\2\6
${}\.{DFS}(\\{nt}\MG\|s,\39\\{nt});{}$\6
\&{return} \\{connected};\6
\4${}\}{}$\2\7
\&{void} \.{DFS}(\&{int} \\{node}${},\39{}$\&{pt\_net} \\{nt})\1\1\2\2\6
${}\{{}$\1\6
\&{int} \|k;\7
\&{if} (\\{connected})\1\5
\&{return};\2\6
\&{if} ${}(\\{node}\E\\{nt}\MG\|t){}$\5
${}\{{}$\1\6
${}\\{connected}\K\T{1};{}$\6
\&{return};\6
\4${}\}{}$\2\6
${}\\{visited}[\\{node}]\K\T{1};{}$\6
${}\|k\K\T{1};{}$\6
\&{while} ${}(\\{list}[\\{node}][\|k]>\T{0}){}$\5
${}\{{}$\1\6
\&{if} ${}(\R\\{visited}[\\{list}[\\{node}][\|k]]){}$\1\5
${}\.{DFS}(\\{list}[\\{node}][\|k],\39\\{nt});{}$\2\6
${}\|k\PP;{}$\6
\4${}\}{}$\2\6
\&{return};\6
\4${}\}{}$\2\par
\U10.\fi

\N{1}{14}Index.

\fi


\inx
\fin
\con
